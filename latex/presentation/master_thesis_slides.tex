\documentclass[notheorems,handout,san serif,hyperref={unicode}]{beamer} %  ,hyperref={bookmarks=false} ,handout
\usepackage{etex}
\usepackage{graphicx,epsfig,picinpar}
\usepackage{epstopdf}
\usepackage[T2A]{fontenc}
\usepackage[utf8]{inputenc} %\usepackage[cp1251]{inputenc}
\usepackage[ukrainian]{babel}
\usepackage{wrapfig}
\usepackage{amsmath,amssymb,amsthm}
\usepackage[mathscr]{eucal}
\usepackage{multicol}
%\usepackage{graphicx,epsfig,psfrag}
\usepackage{tikz}
\usetikzlibrary{calc}
\usetikzlibrary{arrows.meta}
\usetikzlibrary{math,trees}
\usetikzlibrary{shapes}
\usetikzlibrary{positioning,automata}
\usetikzlibrary{arrows,calc}
\usetikzlibrary{decorations.markings}

%\usepackage{multicol}
\usepackage{changepage}
\usepackage{bookmark}

%\usepackage{bbm}
%\usepackage{pgfplots}
%\pgfplotsset{compat=newest,
	%colormap={mycolormap}{color=(white) color=(white) color=(white)}
	%}
%\hypersetup{bookmarksopen=true,bookmarksopenlevel=3}

%\usepackage[usenames]{color}
%\setbeamercolor{section in toc}{fg=darkblue}
%\setbeamercolor{frametitle}{fg=darkblue}



% links
\hypersetup{
	colorlinks=true,
	linkcolor=blue,
	filecolor=magenta,
	urlcolor=cyan,
}



\definecolor{red}{rgb}{1,0.00,0.00}
\definecolor{blue}{rgb}{0,0.08,0.55}
\definecolor{green}{rgb}{0,0.55,0.08}
\definecolor{grey}{rgb}{0.87,0.87,0.87}
\definecolor{white}{rgb}{1,1,1}
\newcommand{\red}[1]{\textcolor{red}{\textbf{#1}}}
\newcommand{\green}[1]{\textcolor{green}{\textbf{#1}}}
\newcommand{\blue}[1]{\textcolor{blue}{\textbf{#1}}}
\newcommand{\white}[1]{\textcolor{white}{#1}}


\newcommand{\Dom}{\text{Dom }}
\newcommand{\Ker}{\text{Ker }}
\newcommand{\Imag}{\text{Im }}
\newcommand{\End}{\text{End}}
\newcommand{\Hom}{\text{Hom}}


\newtheorem{theorem}{Theorem}[section]
\newtheorem{proposition}[theorem]{Proposition}
\newtheorem{corollary}[theorem]{Corrolary}
\newtheorem{question}{Question}

%\newtheorem{lemma}{Лема}[section]
\theoremstyle{definition}
\newtheorem{definition}{Definition}[section]
\newtheorem{example}{Example}[section]
\newtheorem{remark}{Remark}[section]




\title{\Large\textbf{Self-similar actions of virtually abelian groups}\vspace{-0.5cm}}
\author{Зашкольний Давид Олександрович\vspace{-0.7cm}}
\date{26 квітня 2023 р.}

\newenvironment{changemargin}[2]{%
	\begin{list}{}{%
			\setlength{\topsep}{0pt}%
			\setlength{\leftmargin}{#1}%
			\setlength{\rightmargin}{#2}%
			\setlength{\listparindent}{\parindent}%
			\setlength{\itemindent}{\parindent}%
			\setlength{\parsep}{\parskip}%
		}%
		\item[]}{\end{list}}

%%%%%%%%%%%%%%%%%%%%%%%%%%%%%%%%%%%%%%%%%%%%%%%%%%%%%  Customize theorem and proof   %%%%%%%%%%%%%%%%%%%%%%%%%%
\makeatletter
\setbeamertemplate{theorem begin}{%
	%  \begin{\inserttheoremblockenv}% removed
		{%
			\textcolor{blue}{\textbf{\inserttheoremname}.}
			\ifx\inserttheoremaddition\@empty\else\ (\inserttheoremaddition)\fi%
			%    \space% new
		}%
	}
	
	\setbeamertemplate{theorem end}{}
	
	\setbeamertemplate{proof begin}{%
		%  \begin{\inserttheoremblockenv}% removed
			{%
				\vspace{0.1cm}\textcolor{blue}{\textbf{Дов}.}
			}%
		}
		\setbeamertemplate{proof end}{}
		
		\makeatother
		%%%%%%%%%%%%%%%%%%%%%%%%%%%%%%%%%%%%%%%%%%%%%%%%%%%%%%%%%%%%%%%%%%%%%%%%%%%%%%%%%%%%%%%%%%%%%%%
		
		%%%%%%%%%%%%%%%%%%%%%%%%%%%%%%%%%%%%%%%%%%%%%%%%%%%%%  Customize header and footer   %%%%%%%%%%%%%%%%%%%%%%%%%%
		\definecolor{headercolor}{RGB}{0,0,128}
		\definecolor{headertextcolor}{RGB}{200,200,200}
		\definecolor{headercolor2}{RGB}{235,235,255}
		\definecolor{headertextcolor2}{RGB}{0,0,128}
		\setbeamercolor{headerlinecolor}{fg=headertextcolor,bg=headertextcolor2}
		\setbeamercolor{headerlinecolor2}{fg=headertextcolor2,bg=headercolor2}
		\setbeamertemplate{headline}{
			\leavevmode%
			\hbox{%
				\begin{beamercolorbox}[wd=.5\paperwidth,ht=2.25ex,dp=1ex,left]{headerlinecolor}%
					{\hspace{0.5cm}  \insertsection\hfill}   %\parbox{\linewidth}{ текст }
				\end{beamercolorbox}%
				\begin{beamercolorbox}[wd=.5\paperwidth,ht=2.25ex,dp=1ex,right]{headerlinecolor2}%
					{\hfill {Self-similar actions of virtually abelian groups } \hspace{0.5cm}  }
			\end{beamercolorbox}}%
			\vskip0pt%
		}
	

\AtBeginSection[]{
	\begin{frame}
		\vfill
		\centering
		\begin{beamercolorbox}[sep=8pt,center,shadow=true,rounded=true]{title}
			\usebeamerfont{title}\insertsectionhead\par%
		\end{beamercolorbox}
		\vfill
	\end{frame}
}	


\makeatother
\setbeamertemplate{footline}{
	\leavevmode%
	\hbox{%
		\begin{beamercolorbox}[wd=.5\paperwidth,ht=2.25ex,dp=1ex,left]{headerlinecolor}%
			{\hspace{0.5cm}  Зашкольний Д.О. \hfill } %\usebeamerfont{author in head/foot}
		\end{beamercolorbox}%
		\begin{beamercolorbox}[wd=.5\paperwidth,ht=2.25ex,dp=1ex,right]{headerlinecolor2}%
			\usebeamerfont{date in head/foot}\insertshortdate{}\hspace*{2em}
			\insertframenumber{} / \inserttotalframenumber\hspace*{2ex}
	\end{beamercolorbox}}%
	\vskip0pt%
}
\makeatletter
%%%%%%%%%%%%%%%%%%%%%%%%%%%%%%%%%%%%%%%%%%%%%%%%%%%%%%%%%%%%%%%%%%%%%%%%%%%%%%%%%%%%%%%%%%%%%%%


\begin{document}

\begin{frame}
	\titlepage
	Науковий керівник: Бондаренко Євген Володимирович
\end{frame}

\begin{frame}
	\frametitle{Contents}
	\tableofcontents
\end{frame}

\section{Theoretical background}



\subsection{Self-similar actions}

\begin{frame}
	\frametitle{Prerequisites. Self-similar actions}
	
		
	\begin{definition}
		Let $X$ be a finite alphabet. A faithful action of a group $G$ on the set $X^{*}$ is called \textit{self-similar} if for every $g\in G$ and $x\in X$ there exist $y\in X$ and $h\in G$ such that
		$g(xw)=yh(w)$ for all $w\in X^{*}$.
		
		Usually a short notation is used, that originates from another definition using automata: 
		\begin{equation} \label{eq:self-similar definition}
			g \cdot x = y \cdot h	
		\end{equation}
		where $h$ is also noted as $g|_x$ and called a \textit{restriction} of $g$ on $x$.
	\end{definition}

	\vskip 1cm
	
	\begin{proposition} \label{self-similar:subgroup}
		Let $H$ be a subgroup of finite index in the group $G$. If $H$ admits a self-similar action, then $G$ admits a self-similar action.
	\end{proposition}
	
\end{frame}


\begin{frame}
	\frametitle{Problem}
	
	
	\begin{corollary} \label{cor: self-similar actions of cryst}
		Every finitely generated virtually abelian group admits a self-similar action. In particular, every crystallographic group admits a self-similar action.
	\end{corollary}
	
	\vskip 2cm
	
	\begin{question}
		Can we \textbf{describe} every self-similar action of finitely generated virtually abelian groups?
	\end{question}
	
\end{frame}



\subsection{Virtual endomorphisms}

\begin{frame}
	\frametitle{Prerequisites. Virtual endomorphisms}
	
	\begin{definition}
		A \textit{virtual homomorphism} $\phi: G_1 \dashrightarrow G_2$ is a homomorphism of groups $\phi: \Dom \phi \rightarrow G_2$, where $\Dom \phi < G_1$ is a subgroup of finite index called the \textit{domain} of the virtual homomorphism. 
	\end{definition}
		
	\begin{proposition}
		The subgroup $K(\phi)$, also known as \textbf{$\phi$-core}, that is defined as  
		$$
		K(\phi) = \bigcap_{n\ge1}\bigcap_{g \in G} g^{-1} \Dom \phi^n g
		$$
		is the maximal one among the \textbf{normal $\phi$-invariant subgroups} of $G$.
	\end{proposition}
	
	
	\begin{definition}
		A virtual endomorphism $\phi$ is said to be \textit{simple} if it's core is trivial, or in other words $K(\phi) = \{e\}$.
		
	\end{definition}
\end{frame}


\begin{frame}
	
	\frametitle{Prerequisites. Associative virtual endomorphism}
	
	\begin{definition}
		
		The map $\phi_x : G \dashrightarrow G$ defined by the condition
		$$
		g \cdot x = x \cdot \phi_x (g)
		$$ 
		is called \textit{associative virtual endomorphism} of self-similar action $(G, X^*)$.
	\end{definition}
	
	\vskip 1cm
	
	\begin{proposition}
		Let $(G, X^*)$ be a self-similar action of a group $G$ with an arbitrary associative virtual endomorphism $\phi$. 
		
		\begin{enumerate}
			\item If $N$ is a normal subgroup of G, and $N$ is $\phi$-invariant, then $N$ is contained in the kernel of the self-similar action.
			
			\item \textbf{The kernel of self-similar action is equal to the $\phi$-core}.
		\end{enumerate} 
		
	\end{proposition}
\end{frame}





\section{Main results}

\subsection{Self-similar actions of free abelian groups}

\begin{frame}
	\frametitle{Main results. Free abelian groups}
	
	\begin{theorem}\quad
		
		\begin{enumerate}
			
			\item There is a correspondence between virtual endomorphisms $\phi : \mathbb{Z}^n \dashrightarrow \mathbb{Z}^n$ and pairs of integral $n\times n$ matrices $(A_1, A_2)$ such that $\det(A_1) \neq 0$, given by the rule:
			$$
			\pi = (A_1, A_2) \mapsto  \phi_{\pi}: A_1(\mathbb{Z}^n) \rightarrow \mathbb{Z}^n, \quad \phi_\pi (g) = A_2 A_1^{-1} g
			$$
			and two pairs $(A_1, A_2)$ and $(B_1, B_2)$ represent the same $\phi$ if and only if there exists an integral matrix $P$ with $|\det(P)|$ such that $B_1 = A_1 P, \, B_2 = A_2 P$.
			
			\item \label{criterion:free} The $\phi_\pi$ has trivial core if and only if $A = A_2A^{-1}_1$ is invertible and its characteristic polynomial is not divisible by a monic polynomial with integral coefficients. 
			
			
			\item The $\phi_\pi$ is surjective iff $|\det(A_2)| = 1$. 
			
		\end{enumerate}
	\end{theorem}
\end{frame}


\subsection{Self-similar actions of finitely generated abelian groups}


\begin{frame}
	\frametitle{Main results. Finitely generated abelian groups}
	Every finitely generated abelian group $G$ decomposes into a direct sum $G = \mathbb{Z}^n \oplus F$, where $F$ is a finite abelian group. Virtual endomorphisms consist of triplets: 
	\begin{equation} \label{eq:general form of virt_end of finit-gen abelian}
		\phi = \begin{pmatrix}
			A_\phi & 0 \\ 
			B_\phi & C_\phi \\
		\end{pmatrix}
	\end{equation} 
	where $A_\phi \in GL(n, \mathbb{Z})$, $B_\phi \in \Hom(\mathbb{Z}^n, F)$ and $C_\phi$ is a virtual endomorphism on $F$. 
	
	
	\begin{theorem}\label{theorem: self-similar fintely generated}
		Let $\phi : G \dashrightarrow G$ be a virtual endomorphism  of the finitely generated abelian group $G = \mathbb{Z}^n \oplus F$. 
		
		\begin{enumerate}
			\item $\phi$ is simple if and only if $A_\phi$ has no eigenvalue that is algebraic integer and $C_\phi$ is simple as virtual endomorphism on $F$.
			
			\item If $\phi$ is surjective and simple, then $G$ is free abelian.
		\end{enumerate}
		
	\end{theorem}

\end{frame}

\subsection{Self-replicating actions of crystallographic groups}

\begin{frame}
	\frametitle{Main results. Crystallographic groups}
	\begin{definition}
		A \textit{crystallographic group} of dimension $n$ is a cocompact discrete subgroup in $\mathbf{E}(n)$.
	\end{definition}
	
	
	\begin{theorem} \label{theorem: cryst criterion}
		An arbitrary crystallographic group $\Gamma$ that is given by $(G, \mathbb{Z}^n, \alpha)$ admits a self-replicating action if and only if 
		there exists $a = A + t \in \mathbf{A}(n)$ where $A$ is a matrix with rational coefficients such that A has no eigenvalue that is algebraic integer and  $a^{-1} \Gamma a \subset \Gamma$. 
	\end{theorem}
	
	\vskip 0.5cm
		
	\begin{corollary} \label{cor: self-replicating action for cryst}
		Every crystallographic group $\Gamma$ admits a self-replicating action with $a=A+t$ being a scalar matrix $A$ with trivial translation.  
	\end{corollary}
	
	\vskip 0.5cm
	
	
	Combining this result with the results of Bondarenko I. we can formulate the following theorem.
	\begin{theorem}
		A virtually abelian group $G$ admits a self-replicating action if and only if $G$ is crystallographic.
	\end{theorem}
	
	
\end{frame}


\section{Computational experiments}

\begin{frame}
	\frametitle{Computational experiments}
	
	We developed some software to work with self-similar actions of crystallographic groups, since they are objects of practical interest. All the needed code one can find on the GitHub repository: \href{https://github.com/davendiy/master_thesis}{https://github.com/davendiy/master\_thesis}. 
	
	Particularly,
	
	\begin{enumerate}
		\item an algorithm to find the normalizer of a  point group;
		\item an algorithm to build a self-similar action by an arbitrary conjugation in the Affine group and the respective crystallographic group of any dimension; 
		
		\item results of brute-force search of faithful self-similar actions for wallpaper groups and space groups, with alphabet of the minimal possible size.
		
	\end{enumerate}
	
\end{frame}

\begin{frame}
	\vfill
	\centering
	\begin{beamercolorbox}[sep=8pt,center,shadow=true,rounded=true]{title}
		\usebeamerfont{title} Thank you for attention!
	\end{beamercolorbox}
	\vfill
\end{frame}


\end{document}


