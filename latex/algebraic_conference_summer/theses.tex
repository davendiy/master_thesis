\documentclass[12pt]{article}
\usepackage{a4wide}
\usepackage{amsmath,amssymb,amsthm}
\usepackage[mathscr]{eucal}
\usepackage[shortlabels]{enumitem}
\usepackage{tikz}
\usetikzlibrary{positioning,automata}
\usetikzlibrary{arrows,calc}
%\input{arrowsnew}
%\usepackage{graphics}
%\usepackage{hyperref}
%\usepackage{epsfig}
%\usepackage{psfrag}



\newtheorem{theorem}{Theorem}
\newtheorem{lemma}{Lemma}

\newtheorem{proposition}[theorem]{Proposition}
\newtheorem{corollary}{Corollary}[theorem]
\newtheorem{conjecture}{Conjecture}

\theoremstyle{definition}
\newtheorem{definition}{Definition}
\newtheorem{example}{Example}
\newtheorem{problem}{Problem}
\newtheorem{question}{Question}
\newtheorem{remark}{Remark}
\newtheorem{algorithm}{Scheme}

\newcommand{\Dom}{\text{Dom }}
\newcommand{\Ker}{\text{Ker }}
\newcommand{\Imag}{\text{Im }}
\newcommand{\End}{\text{End}}
\newcommand{\Hom}{\text{Hom}}


\author{Ievgen Bondarenko and David Zashkolny}
\title{\Large\textbf{Self-similar actions of virtually abelian groups}\vspace{-0.5cm}}

\begin{document}
	\maketitle
	
	Self-similar group actions are special actions on the spaces of words that reflect the self-similarity of the space. Self-similar group action naturally arise in many areas of mathematics: dynamical systems, fractal geometry, algebraic topology, automata theory. For the last twenty years self-similar actions were studied for many classes of groups: abelian, nilpotent, solvable, free and linear groups, arithmetic groups. 
	
	Self-replicating actions is the special case of self-similar actions. There is a convenient algebraic criterion: a group $G$ admits a self-similar action if and only if there is a virtual homomorphism $\phi : H \rightarrow G$ (i.e. $H < G$ is a subgroup of finite index) and the $\phi$-core is trivial. Similarly, a group admits a self-replicating action if there is such a surjective $\phi$. A self-similar action associated to $\phi$ is obtained by a certain iterated construction. 
	
	Every finitely generated virtually abelian group admits a self-similar action, yet not every one does self-replicating action. Nekrashevych-Sidki \cite{Nekrashevych} showed that if an abelian group has such an action, then it is free. Thus, we consider a generalization of this result to virtually abelian groups.
	
	A crystallographic group of dimension $n$ is a discrete cocompact group of isometries of $\mathbb{R}^n$. Recalling the Bieberbach theorem, every isomorphism of crystallographic group is in fact a conjugation by an element $a = (A, t)$ from the affine group $A(n)$. We have proven the following theorems: 
	
	\begin{theorem}
		Every crystallographic group $G$ admits a self-replicating action, that is generated by a virtual endomorphism $\phi(g) = a^{-1}ga$, where $a = (A, t)$ is an affine transformation with trivial translation $t = 0$ and a scalar matrix $A$.  
	\end{theorem}
	
	\begin{theorem}
		A virtually abelian group admits self-replicating action if and only if it is crystallographic.
	\end{theorem}
	
	Also, we designed and implemented an algorithm to create a self-similar action by the virtual endomorphisms. Using it, we found self-replicating actions with minimal alphabet for every crystallographic group of dimensions 2 and 3. 
	
	\begin{thebibliography}{9}
		\bibitem{Nekrashevych} Nekrashevych V. Self-similar groups. -- Providence: Mathematical Surveys and Monographs, Vol.117, American Mathematical Society, 2005, 231 pages.
	\end{thebibliography}
\end{document}